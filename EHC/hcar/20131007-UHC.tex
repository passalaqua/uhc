% UHCUtrechtHaskellCompiler-AU.tex
\begin{hcarentry}[section]{UHC, Utrecht Haskell Compiler}
\label{uhc}
\label{ehc}
\report{Atze Dijkstra}%05/12
\status{active development}
\participants{many others%Jeroen Fokker,
% Doaitse Swierstra,
% Arie Middelkoop,
% Luc\'ilia Camar\~{a}o de Figueiredo,
% Carlos Camar\~{a}o de Figueiredo,
% Andres L\"oh, Jos\'e~Pedro Magalh\~{a}es,
% Vincent van Oostrum, Clemens Grabmayer, Jan Rochel,
% Tom Lokhorst, Jeroen Leeuwestein, Atze van der Ploeg, Paul van der Ende, Calin Juravle, Levin Fritz
}
\makeheader

UHC is the Utrecht Haskell Compiler, supporting almost all Haskell98 features and most of Haskell2010, plus
experimental extensions.

\paragraph{Status}

Current work is on a strictness analyser (Augusto Pasalaqua) and
incrementality of analysis via the Attribute Grammar system used to construct UHC (Jeroen Bransen).

\paragraph{Background.}

UHC actually is a series of compilers of which the last is UHC, plus
infrastructure for facilitating experimentation and extension.
The distinguishing features for dealing with the complexity of the compiler and for experimentation are
(1) its stepwise organisation as a series of increasingly more complex standalone compilers,
the use of DSL and tools for its (2) aspectwise organisation (called Shuffle) and
(3) tree-oriented programming (Attribute Grammars, by way of the
Utrecht University Attribute Grammar (UUAG) system~\cref{uuag}.

\FurtherReading
\begin{compactitem}
\item UHC Homepage:
\url{http://www.cs.uu.nl/wiki/UHC/WebHome}

\item UHC Github repository:
\url{https://github.com/UU-ComputerScience/uhc}

\item UHC Javascript backend:
\url{http://uu-computerscience.github.com/uhc-js/}

\item Attribute grammar system:
\url{http://www.cs.uu.nl/wiki/HUT/AttributeGrammarSystem}

\end{compactitem}
\end{hcarentry}
